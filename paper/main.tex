%%%%%%%%%%%%%%%%%%%%%%%%%%%%%%%%%%%%%%%%%%%%%%%%%%%%%%%%%%%%%%%%%%%%%%
%%                                                                 %%
%% Please do not use \input{...} to include other tex files.       %%
%% Submit your LaTeX manuscript as one .tex document.              %%
%%                                                                 %%
%% All additional figures and files should be attached             %%
%% separately and not embedded in the \TeX\ document itself.       %%
%%                                                                 %%
%%%%%%%%%%%%%%%%%%%%%%%%%%%%%%%%%%%%%%%%%%%%%%%%%%%%%%%%%%%%%%%%%%%%%

% Math and Physical Sciences Numbered Reference Style
\documentclass[pdflatex,sn-mathphys-num]{sn-jnl}

% Standard Packages
\usepackage{algorithm}
\usepackage{algorithmicx}
\usepackage{algpseudocode}
\usepackage{amsmath,amssymb,amsfonts}
\usepackage{amsthm}
\usepackage{booktabs}
\usepackage[title]{appendix}
\usepackage{graphicx}
\usepackage{listings}
\usepackage{manyfoot}
\usepackage{mathrsfs}
\usepackage{multirow}
\usepackage{textcomp}
\usepackage{xcolor}

% Other
\geometry{margin=1in, bindingoffset=0in, asymmetric=false}
\raggedbottom
\newcommand{\df}[2]{\displaystyle\frac{#1}{#2}}

\begin{document}

\title[Article Title]{Optimizing irrigation and fertilizer strategy using a crop growth model with delayed nutrient absorption dynamics}

%======================================================================================%
%% TARGET JOURNAL: Computers and Electronics in Agriculture
%% https://www.sciencedirect.com/journal/computers-and-electronics-in-agriculture/publish/guide-for-authors

% Paper A (Model + GA):
% Given a nonlinear, delay-affected plant growth system with resource penalties, can 
% global optimization discover non-intuitive but high-performing control strategies?

% Paper B (MPC):
% Given that same system, can we close the loop and recover near-optimal performance
% under uncertainty and disturbances?
%======================================================================================%

\author*[1]{\fnm{Carla J.} \sur{Becker}}\email{carlabecker@berkeley.edu, ORCID: 0009-0004-0646-6386}

\author[1]{\fnm{Tarek I.} \sur{Zohdi}}\email{zohdi@berkeley.edu, ORCID: 0000-0002-0844-3573}
%\equalcont{These authors contributed equally to this work.}

\affil*[1]{\orgdiv{Department of Mechanical Engineering}, \orgname{University of California, Berkeley}, \orgaddress{\street{6141 Etcheverry Hall}, \city{Berkeley}, \postcode{94720}, \state{California}, \country{United States of America}}}

%======================================================================================%
% ABSTRACT
%======================================================================================%
\abstract{Rising production costs and declining commodity prices have motivated farmers to seek computational tools for optimizing resource application. This paper presents a generalized, coupled ordinary differential equation (ODE) model for crop growth that captures the nonlinear dynamics of plant development under varying environmental conditions and fertilizer and irrigation strategy. The model tracks five state variables---plant height, leaf area, number of leaves, flower size, and fruit biomass---each governed by logistic growth with time-varying growth rates and carrying capacities. These parameters are modulated by nutrient factors that quantify how well actual water, fertilizer, temperature, and solar radiation levels match expected values. To capture the delayed physiological response of plants to resource inputs, we employ finite impulse response (FIR) convolution with Gaussian kernels, where different temporal spreads represent the distinct metabolic timescales for each input type. Cumulative divergence from expected nutrient levels is tracked using exponential moving average filters, providing the model with memory of past stress events. Given this nonlinear, delay-affected system, we employ a genetic algorithm (GA) to discover optimal irrigation and fertilizer strategies that maximize crop yield while minimizing input costs. The GA searches over application frequency and amount for both irrigation and fertilizer, evaluating candidate strategies through full-season simulations. We demonstrate the framework using corn grown in Fairfax, Iowa, with historical weather data. Results show the GA identifies non-intuitive strategies that outperform conventional uniform application schedules, achieving higher net revenue through strategic timing of resource inputs.}

\keywords{precision agriculture, resource optimization, crop growth model, genetic algorithm, cumulative stress tracking}
\maketitle

%======================================================================================%
% INTRODUCTION
%======================================================================================%
\section{Introduction}\label{sec:intro}

The agriculture sector in the United States faces significant challenges as the number of farms declines and the cost of farming continues to rise \cite{usda-farm-income}. Rising production expenses for equipment, seeds, and labor, coupled with elevated interest rates and declining commodity prices, have made farming increasingly expensive. To navigate this challenging landscape, farmers are employing strategies such as cost management and operational optimization. One promising approach is to use modeling and simulation to optimize farm operations without substantial capital investment.

Mathematical modeling of crop growth has a rich history, with models ranging from simple empirical relationships to complex mechanistic simulations. Logistic growth models, first proposed by Verhulst in 1838, remain widely used due to their interpretability and ability to capture resource-limited growth dynamics \cite{verhulst1838}. More sophisticated crop models such as DSSAT \cite{jones2003dssat}, APSIM \cite{holzworth2014apsim}, and WOFOST \cite{wofost} simulate detailed physiological processes but require extensive parameterization and may be computationally expensive for optimization applications. In contrast, reduced-order models that capture essential dynamics while remaining tractable for optimization have gained attention in precision agriculture \cite{precision-ag-review}.

Optimization of irrigation and fertilizer application has been studied using various approaches, including linear and nonlinear programming \cite{irrigation-optimization}, dynamic programming \cite{dp-irrigation}, and metaheuristic algorithms \cite{ga-agriculture}. Genetic algorithms (GAs) are particularly well-suited for this domain because they can handle nonlinear, non-convex objective functions and do not require gradient information \cite{goldberg1989ga}. Previous work has applied GAs to irrigation scheduling \cite{ga-irrigation-scheduling} and fertilizer optimization \cite{ga-fertilizer}, but these studies often use simplified plant response models that do not capture the delayed, cumulative effects of resource application.

This paper presents a generalized, coupled ordinary differential equation (ODE) model for crop growth that addresses these limitations. The model captures: (1) nonlinear logistic growth with state-dependent carrying capacities, (2) delayed absorption of water, fertilizer, temperature, and solar radiation inputs through finite impulse response (FIR) convolution, (3) cumulative stress tracking via exponential moving average (EMA) filters, and (4) coupling between vegetative and reproductive growth stages. The model is designed to enable global optimization under delayed, resource-coupled dynamics---a regime where even well-established management practices may benefit from computational refinement due to the sheer number of possible scheduling combinations.

We demonstrate the framework by optimizing irrigation and fertilizer strategies for corn, the most widely planted crop in the United States by acreage. Using a genetic algorithm, we search for strategies that maximize net revenue (crop value minus input costs) over a growing season. The approach is validated using historical weather data from Fairfax, Iowa, a representative location in the Corn Belt.


%======================================================================================%
% MODELING
%======================================================================================%
\section{Generalized, Coupled-ODE Crop Model}\label{sec:modeling}

The proposed model tracks five state variables representing key aspects of plant development: plant height $h$ (m), leaf area per leaf $A$ (m$^{2}$), number of leaves $N$, flower size $c$ (number of spikelets), and fruit biomass $P$ (kg). Each state variable follows logistic growth dynamics with time-varying growth rates and carrying capacities that depend on environmental conditions and resource availability.

The model receives four input signals: water from irrigation $w$ (inches), fertilizer application $f$ (lbs), ambient temperature $T$ ($^{\circ}$C), and solar radiation $R$ (W/m$^{2}$). Precipitation $S$ (inches) is added to irrigation to obtain total water input. Temperature and radiation data are obtained from the National Solar Radiation Database (NSRDB) \cite{nsrdb}, while precipitation data comes from NOAA historical records \cite{noaa-precip}.

%------------------------------------%
% DELAYED ABSORPTION
%------------------------------------%
\subsection{Delayed Absorption via FIR Convolution}\label{subsec:fir-conv}

Plants do not immediately process applied nutrients; instead, there is a physiologically-mediated delay between application and utilization. We model this delayed absorption using finite impulse response (FIR) convolution with Gaussian kernels.

If cumulative nutrient uptake follows a sigmoid trajectory---with slow initial uptake due to transport lag, rapid increase once metabolic pathways activate, and eventual saturation---then instantaneous absorption rate follows a bell-shaped curve. A Gaussian kernel is the least assumptive choice for a bell curve, requiring only two parameters: the temporal spread $\sigma$ (characterizing absorption duration) and the peak delay $\mu$.

Given only the temporal spread $\sigma$ for each nutrient type, we determine $\mu$ such that 95\% of the kernel mass lies within $[0, 2\mu]$. This requires solving
\begin{equation}
\text{erf}\left(\df{\mu}{\sigma\sqrt{2}}\right) = 0.95,
\label{eqn:erf}
\end{equation}
which yields $\mu \approx 1.96\sigma$. The Gaussian FIR kernel is then
\begin{equation}
g[k] = \df{1}{\sqrt{2\pi\sigma^{2}}} \exp\left\{ -\df{1}{2}\df{(k-\mu)^{2}}{\sigma^{2}} \right\}.
\label{eqn:gaussian-kernel}
\end{equation}

The FIR horizon $L^{*}$ is chosen as the minimum length capturing 95\% of the kernel mass:
\begin{equation}
L^{*} = \min_{L} \left\{ L : \df{\sum_{k=0}^{L-1} g[k]}{\sum_{k=0}^{K-1} g[k]} \geq 0.95 \right\}
\label{eqn:L-star}
\end{equation}
where $K$ is the simulation length in hours.

Different nutrients have different metabolic timescales. We use $\sigma_{w} = 30$ hours for water (rapid uptake), $\sigma_{f} = 300$ hours for fertilizer (slow uptake reflecting root absorption dynamics), and $\sigma_{T} = \sigma_{R} = 30$ hours for temperature and radiation (immediate physiological effects with short memory).

%------------------------------------%
% CUMULATIVE STRESS TRACKING
%------------------------------------%
\subsection{Cumulative Stress Tracking via EMA Filtering}\label{subsec:ema-smooth}

While FIR convolution captures delayed absorption, plants also accumulate stress from sustained deviations from optimal conditions. We model this cumulative effect using exponential moving average (EMA) filters, which are equivalent to first-order infinite impulse response (IIR) systems.

The EMA filter with memory parameter $\beta \in [0,1)$ has the recursive form:
\begin{equation}
y[k] = (1 - \beta)x[k] + \beta y[k-1]
\label{eqn:iir-filter}
\end{equation}
where larger $\beta$ values correspond to longer memory (slower response to changes). This formulation preserves constant signals ($x[k] = c$ implies $y[k] \rightarrow c$) while smoothing transient fluctuations.

%------------------------------------%
% NUTRIENT FACTORS
%------------------------------------%
\subsection{Nutrient Factor Calculation}\label{subsec:nutrient-factors}

We now describe the complete transformation pipeline that converts raw input signals into nutrient factors $\nu \in [0,1]$ that modulate plant growth. Using fertilizer as an example (the same process applies to water, temperature, and radiation):

\textbf{Step 1: Delayed absorption.} Convolve the raw fertilizer signal $f[k]$ with the Gaussian FIR kernel:
\begin{equation}
\bar{f}[k] = \sum_{n=0}^{L_{f}-1} g_{f}[n]f[k-n]
\label{eqn:delayed-trans}
\end{equation}

\textbf{Step 2: Cumulative absorption.} Compute the running sum of absorbed fertilizer:
\begin{equation}
F[k] = \sum_{n=0}^{k} \bar{f}[n]
\label{eqn:cum-trans}
\end{equation}

\textbf{Step 3: Instantaneous anomaly.} Compare actual cumulative absorption to expected levels:
\begin{equation}
\delta_{f}[k] = \left| \df{k \cdot f_{\text{typ}} - F[k]}{k \cdot f_{\text{typ}} + \epsilon} \right|
\label{eqn:anomaly-trans}
\end{equation}
where $f_{\text{typ}}$ is the typical hourly fertilizer level the plant ``expects'' and $\epsilon$ is a small constant preventing division by zero.

\textbf{Step 4: Cumulative divergence.} Apply EMA smoothing to track sustained anomalies:
\begin{equation}
\Delta_{f}[k] = \beta_{\Delta} \Delta_{f}[k-1] + (1 - \beta_{\Delta}) \delta_{f}[k]
\label{eqn:cum-div-trans}
\end{equation}
where $\beta_{\Delta} = 0.95$ provides long memory.

\textbf{Step 5: Nutrient factor.} Convert divergence to a stress factor via exponential decay with additional EMA smoothing:
\begin{equation}
\nu_{f}[k] = \beta_{\nu} \nu_{f}[k-1] + (1 - \beta_{\nu}) \exp\{ -\alpha\Delta_{f}[k] \}
\label{eqn:nut-fac-trans}
\end{equation}
where $\alpha = 3$ ensures $\nu \approx 0.05$ when $\Delta = 1$ (complete divergence from expected levels), and $\beta_{\nu} = 0.05$.

The nutrient factor $\nu_{f}[k]$ equals 1 when fertilizer application perfectly matches expected levels and decays toward 0 under sustained over- or under-application. This captures the intuition that plants are resilient to brief deviations but suffer cumulative damage from prolonged stress.

%------------------------------------%
% INPUT EFFECTS
%------------------------------------%
\subsection{Effects of Inputs on Plant Growth}\label{subsec:input-effects}

Different inputs affect different aspects of plant growth. Tables \ref{tab:fert-irr-effects} and \ref{tab:temp-rad-effects} summarize these relationships based on agronomic literature \cite{corn-nitrogen,corn-irrigation,corn-temperature}.

\begin{table}[h]
\centering
\begin{tabular}{|c|c|c|c|c|}
\hline
State variable &
\multicolumn{1}{|p{2cm}|}{\centering Irrigation on \\ growth rate} &
\multicolumn{1}{|p{2cm}|}{\centering Fertilizer on \\ growth rate} &
\multicolumn{1}{|p{2cm}|}{\centering Irrigation on \\ capacity}    &
\multicolumn{1}{|p{2cm}|}{\centering Fertilizer on \\ capacity}    \\
\hline\hline
Plant height $h$     & $\sim$ & +      & $\sim$ & +      \\
Leaf area $A$        & $\sim$ & +      & +      & +      \\
Number of leaves $N$ & $\sim$ & $\sim$ & $\sim$ & $\sim$ \\
Flower size $c$      & $\sim$ & $\sim$ & +      & $\sim$ \\
Fruit biomass $P$    & $\sim$ & $\sim$ & +      & +      \\
\hline
\end{tabular}
\caption{Effects of irrigation and fertilizer on growth dynamics. ``+'' indicates positive effect, ``$\sim$'' indicates a negligible effect.}
\label{tab:fert-irr-effects}
\end{table}

\begin{table}[h]
\centering
\begin{tabular}{|c|c|c|c|c|}
\hline
State variable &
\multicolumn{1}{|p{2cm}|}{\centering Temp.\ on \\ growth rate} &
\multicolumn{1}{|p{2cm}|}{\centering Temp.\ on \\ capacity} &
\multicolumn{1}{|p{2cm}|}{\centering Radiation on \\ growth rate} &
\multicolumn{1}{|p{2cm}|}{\centering Radiation on \\ capacity} \\
\hline\hline
Plant height $h$     & +      & +  & +      & +  \\
Leaf area $A$        & +      & +  & +      & +  \\
Number of leaves $N$ & $\sim$ & +  & $\sim$ & +  \\
Flower size $c$      & --     & -- & --     & -- \\
Fruit biomass $P$    & +      & +  & +      & +  \\
\hline
\end{tabular}
\caption{Effects of temperature and solar radiation on growth dynamics. ``+'' indicates positive effect, ``$\sim$'' indicates a negligible effect, ``--'' indicates a negative effect. For flower size, excess heat and radiation reduce flower development, hence negative effects.}
\label{tab:temp-rad-effects}
\end{table}

%------------------------------------%
% DYNAMICS
%------------------------------------%
\subsection{Growth Dynamics}\label{subsec:dynamics}

Each state variable follows logistic growth with time-varying parameters modulated by nutrient factors. The general form is:
\begin{equation}
\df{dx}{dt} = \hat{a}_{x}(t) \cdot x(t) \left( 1 - \df{x(t)}{\hat{k}_{x}(t)} \right)
\label{eqn:logistic-general}
\end{equation}
where $\hat{a}_{x}(t)$ is the effective growth rate and $\hat{k}_{x}(t)$ is the effective carrying capacity, both functions of the nutrient factors.

The effective parameters are computed as geometric means of the relevant nutrient factors, reflecting multiplicative rather than additive effects. This choice is motivated by the observation that growth rates compound over time, making geometric averaging appropriate \cite{geometric-mean-growth}.

\textbf{Plant height} responds to fertilizer, temperature, and radiation:
\begin{equation}
\hat{a}_{h}(t) = a_{h}(\nu_{f}\nu_{T}\nu_{R})^{1/3}, \quad \hat{k}_{h}(t) = k_{h}(\nu_{f}\nu_{T}\nu_{R})^{1/3}
\label{eqn:hats-for-h}
\end{equation}

\textbf{Leaf area} additionally depends on water and is coupled to height:
\begin{equation}
\hat{a}_{A}(t) = a_{A}(\nu_{f}\nu_{T}\nu_{R})^{1/3}, \quad \hat{k}_{A}(t) = k_{A}\left(\nu_{w}\nu_{f}\nu_{T}\nu_{R}\df{\hat{k}_{h}}{k_{h}}\right)^{1/5}
\label{eqn:hats-for-A}
\end{equation}

\textbf{Number of leaves} depends only on temperature and radiation through the carrying capacity:
\begin{equation}
\hat{a}_{N}(t) = a_{N}, \quad \hat{k}_{N}(t) = k_{N}(\nu_{T}\nu_{R})^{1/2}
\label{eqn:hats-for-N}
\end{equation}

\textbf{Flower size} (spikelet count) exhibits inverse dependence on temperature and radiation---excess heat and light reduce flowering:
\begin{equation}
\hat{a}_{c}(t) = a_{c}\left(\df{1}{\nu_{T}}\df{1}{\nu_{R}}\right)^{1/2}, \quad \hat{k}_{c}(t) = k_{c}\left(\nu_{w}\df{1}{\nu_{T}}\df{1}{\nu_{R}}\right)^{1/3}
\label{eqn:hats-for-c}
\end{equation}

\textbf{Fruit biomass} depends on all inputs and is coupled to vegetative growth:
\begin{equation}
\hat{a}_{P}(t) = a_{P}\left(\df{1}{\nu_{T}}\df{1}{\nu_{R}}\right)^{1/2}, \quad \hat{k}_{P}(t) = k_{P}\left(\nu_{w}\nu_{f}\nu_{T}\nu_{R}\df{\hat{k}_{h}}{k_{h}}\df{\hat{k}_{A}}{k_{A}}\df{\hat{k}_{c}}{k_{c}}\right)^{1/7}
\label{eqn:hats-for-P}
\end{equation}

The coupling terms $\hat{k}_{h}/k_{h}$, $\hat{k}_{A}/k_{A}$, and $\hat{k}_{c}/k_{c}$ encode physiological dependencies: taller plants with more leaf area can support larger fruit, while larger tassels (more spikelets) may compete with ear development.

%------------------------------------%
% PARAMETERS
%------------------------------------%
\subsection{Model Parameters}\label{subsec:parameters}

The baseline growth rates and carrying capacities are crop-specific parameters that can be estimated from field data or literature values. For corn, we use the values in Table \ref{tab:model-params}, calibrated to match typical development timelines where plants reach full vegetative size around 65--70 days after sowing and grain fill completes around 125 days \cite{corn-growth-stages}.

\begin{table}[h]
\centering
\begin{tabular}{|c|c|c|c|c|}
\hline
State & Growth rate & Carrying capacity & Initial condition & Units \\
\hline\hline
Height $h$ & $a_{h} = 0.010$ hr$^{-1}$ & $k_{h} = 3.0$ & $h_{0} = 0.001$ & m \\
Leaf area $A$ & $a_{A} = 0.0105$ hr$^{-1}$ & $k_{A} = 0.65$ & $A_{0} = 0.001$ & m$^{2}$ \\
Leaves $N$ & $a_{N} = 0.011$ hr$^{-1}$ & $k_{N} = 20$ & $N_{0} = 0.001$ & count \\
Spikelets $c$ & $a_{c} = 0.010$ hr$^{-1}$ & $k_{c} = 1000$ & $c_{0} = 0.001$ & count \\
Fruit $P$ & $a_{P} = 0.005$ hr$^{-1}$ & $k_{P} = 0.25$ & $P_{0} = 0.001$ & kg \\
\hline
\end{tabular}
\caption{Baseline model parameters for corn. Growth rates are per hour; initial conditions are set to $k_{x}/K$ where $K \approx 2900$ is the season length in hours.}
\label{tab:model-params}
\end{table}


%======================================================================================%
% SIMULATION
%======================================================================================%
\section{Simulation}\label{sec:simulation}

The logistic ODE admits a closed-form solution, enabling exact time-stepping without numerical integration error. Given state $x(t)$ at time $t$, the state at $t + \Delta t$ is:
\begin{equation}
x(t + \Delta t) = \df{\hat{k}_{x}(t)}{1 + \left(\df{\hat{k}_{x}(t)}{x(t)} - 1\right)\exp(-\hat{a}_{x}(t)\Delta t)}
\label{eqn:closed-form-logistic}
\end{equation}
where $\hat{a}_{x}(t)$ and $\hat{k}_{x}(t)$ are treated as constant over the time step. This closed-form approach is more accurate than forward Euler integration and avoids instability issues that can arise with explicit methods at larger time steps.

We simulate the growing season at hourly resolution ($\Delta t = 1$ hour), yielding approximately 2900 time steps for a typical corn season (late April to early October). At each step, we: (1) update the nutrient factors based on cumulative inputs and divergences, (2) compute effective growth rates and carrying capacities, and (3) advance each state variable using Equation \ref{eqn:closed-form-logistic}.

%======================================================================================%
% GENETIC ALGORITHM
%======================================================================================%
\section{Optimization via Genetic Algorithm}\label{sec:ga}

Given the nonlinear, delay-affected dynamics of the crop model, gradient-based optimization is challenging. The delayed effects of inputs create a non-convex landscape with potentially many local optima. We therefore employ a genetic algorithm (GA), a population-based metaheuristic inspired by natural selection that can effectively explore complex search spaces \cite{goldberg1989ga}.

%------------------------------------%
% DESIGN VARIABLES
%------------------------------------%
\subsection{Decision Variables}\label{subsec:design-vars}

Each candidate solution encodes a complete irrigation and fertilization strategy as a four-dimensional vector:
\begin{equation}
\mathbf{u} = \begin{bmatrix} u_{1} \\ u_{2} \\ u_{3} \\ u_{4} \end{bmatrix} = \begin{bmatrix} \text{irrigation frequency (hours)} \\ \text{irrigation amount (inches)} \\ \text{fertilizer frequency (hours)} \\ \text{fertilizer amount (lbs)} \end{bmatrix}
\label{eqn:design-vars}
\end{equation}

The frequencies specify application intervals: $u_{1} = 168$ means irrigate every 168 hours (weekly). The amounts specify the quantity applied at each event. This parameterization assumes regular, periodic application---a simplification that captures common agricultural practice while keeping the search space tractable.

%------------------------------------%
% OBJECTIVE FUNCTION
%------------------------------------%
\subsection{Objective Function}\label{subsec:objective}

The objective is to maximize net revenue, defined as crop value minus input costs. We formulate this as a minimization problem by negating revenue:

\begin{equation}
\text{Cost}(\mathbf{u}) = -\text{Revenue}(\mathbf{u}) = -(\text{Crop Value} - \text{Input Costs})
\label{eqn:cost-function}
\end{equation}

The crop value depends on final plant state at harvest:
\begin{equation}
\text{Crop Value} = \omega_{h}h[K] + \omega_{A}A[K] + \omega_{P}P[K]
\label{eqn:crop-value}
\end{equation}
where $K$ is the final time step and $\omega_{h}$, $\omega_{A}$, $\omega_{P}$ are economic weights (dollars per unit) for height, leaf area, and fruit biomass respectively.

The input costs accumulate over the season:
\begin{equation}
\text{Input Costs} = \omega_{w}\sum_{k=0}^{K}w[k] + \omega_{f}\sum_{k=0}^{K}f[k]
\label{eqn:input-costs}
\end{equation}
where $\omega_{w}$ and $\omega_{f}$ are costs per unit of irrigation and fertilizer.

For corn, the economic weights are derived from market prices and typical yields (Table \ref{tab:economic-weights}). The fruit biomass weight dominates, reflecting that grain yield is the primary economic output.

\begin{table}[h]
\centering
\begin{tabular}{|c|c|c|}
\hline
Parameter & Value & Derivation \\
\hline\hline
$\omega_{w}$ & \$2.00/inch & Typical irrigation cost \\
$\omega_{f}$ & \$0.61/lb & Weighted NPK cost \\
$\omega_{h}$ & \$35/m & Silage value proxy \\
$\omega_{A}$ & \$215/m$^{2}$ & Silage value proxy \\
$\omega_{P}$ & \$4,450/kg & \$4/bushel $\times$ plant density \\
\hline
\end{tabular}
\caption{Economic weights for the corn objective function. The fruit biomass weight accounts for approximately 28,350 plants per acre at \$0.157/kg.}
\label{tab:economic-weights}
\end{table}

%------------------------------------%
% GA ALGORITHM
%------------------------------------%
\subsection{Algorithm Description}\label{subsec:ga-algorithm}

The GA maintains a population of $M$ candidate solutions and iteratively improves them through selection, crossover, and mutation over $G$ generations. Algorithm \ref{alg:ga} presents the complete procedure.

\begin{algorithm}[h]
\caption{Genetic Algorithm for Input Optimization}
\label{alg:ga}
\begin{algorithmic}[1]
\State \textbf{Input:} Population size $M$, parents $P$, children $C$, generations $G$, bounds $[\mathbf{u}_{\min}, \mathbf{u}_{\max}]$
\State \textbf{Output:} Best solution $\mathbf{u}^{*}$
\State
\State Initialize population $\{\mathbf{u}^{(1)}, \ldots, \mathbf{u}^{(M)}\}$ uniformly in bounds
\State Evaluate $\text{Cost}(\mathbf{u}^{(i)})$ for all $i$ via full-season simulation
\State Sort population by cost (ascending)
\State $\text{stagnation} \gets 0$
\State
\For{$g = 1$ to $G$}
    \State \textbf{Selection:} Keep top $P$ members as parents
    \State
    \State \textbf{Crossover:} Generate $C$ children
    \For{$j = 1$ to $C$}
        \State Select parents $\mathbf{u}^{(a)}$, $\mathbf{u}^{(b)}$ randomly from top $P$
        \If{$\text{stagnation} < 10$}
            \State $\phi \sim \text{Uniform}(0, 1)$
        \Else
            \State $\phi \sim \text{Uniform}(-1, 2)$ \Comment{Aggressive exploration}
        \EndIf
        \State $\mathbf{u}^{(\text{child})} \gets \phi \cdot \mathbf{u}^{(a)} + (1 - \phi) \cdot \mathbf{u}^{(b)}$
        \State Clip to bounds
    \EndFor
    \State
    \State \textbf{Fill remaining:} Generate $M - P - C$ random members
    \State Evaluate costs for new members
    \State Sort population by cost
    \State
    \State \textbf{Stagnation check:}
    \If{$|\text{Cost}^{(g)} - \text{Cost}^{(g-1)}| < 0.01$}
        \State $\text{stagnation} \gets \text{stagnation} + 1$
    \Else
        \State $\text{stagnation} \gets 0$
    \EndIf
\EndFor
\State
\State \Return $\mathbf{u}^{(1)}$ (best member)
\end{algorithmic}
\end{algorithm}

\textbf{Selection.} After each generation, population members are ranked by cost. The top $P$ members survive as ``parents'' for the next generation. This selection ensures the best solutions are never lost.

\textbf{Crossover.} New ``children'' are created by blending two parent solutions. For each child, we randomly select two parents and compute a weighted average:
\begin{equation}
\mathbf{u}^{(\text{child})} = \phi \cdot \mathbf{u}^{(a)} + (1 - \phi) \cdot \mathbf{u}^{(b)}
\label{eqn:crossover}
\end{equation}
where $\phi \sim \text{Uniform}(0, 1)$ under normal operation. This crossover can produce children anywhere along the line segment connecting the parents, enabling smooth exploration of the search space.

\textbf{Mutation and Diversity.} To maintain population diversity and escape local optima, the remaining $M - P - C$ population slots are filled with randomly generated solutions. Additionally, if the best cost stagnates (changes by less than 0.01) for 10 consecutive generations, we switch to aggressive crossover with $\phi \sim \text{Uniform}(-1, 2)$. This allows children to lie outside the convex hull of their parents, promoting exploration of new regions.

\textbf{Default Parameters.} We use $M = 128$ members, $P = 16$ parents, $C = 16$ children, and $G = 100$ generations. The large population relative to generations ensures diversity for exploration while preventing premature convergence.


%======================================================================================%
% CASE STUDY
%======================================================================================%
\section{Case Study: Corn in Iowa}\label{sec:case-study}

We demonstrate the framework using corn, the most widely planted crop in the United States with over 90 million acres harvested annually \cite{fsa-acreage}. The case study uses historical weather data from Fairfax, Iowa (41.76$^{\circ}$N, 91.87$^{\circ}$W), a representative location in the Corn Belt (USDA climate zones 4b--5b).

%------------------------------------%
% SCENARIO SETUP
%------------------------------------%
\subsection{Scenario Configuration}\label{subsec:scenario}

The simulation covers a typical growing season from late April to early October (approximately 2900 hours). Environmental inputs are:

\begin{itemize}
\item \textbf{Temperature and radiation:} Hourly data from NSRDB for Fairfax, IA. Mean temperature is 22.8$^{\circ}$C; mean solar radiation is 580 W/m$^{2}$.
\item \textbf{Precipitation:} Daily data from NOAA, interpolated to hourly resolution.
\item \textbf{Typical nutrient expectations:} Based on agronomic recommendations \cite{corn-nutrient-management}, the model expects 28 inches of water and 355 lbs of NPK fertilizer over the season ($w_{\text{typ}} \approx 0.01$ in/hr, $f_{\text{typ}} \approx 0.12$ lb/hr).
\end{itemize}

Table \ref{tab:corn-development} summarizes expected corn development timelines used to calibrate model parameters.

\begin{table}[h]
\centering
\begin{tabular}{|c|c|c|c|}
\hline
State variable & Days to maturity & Hours to maturity & Typical final value \\
\hline\hline
Plant height $h$     & 65--70 & 1560--1680 & 2.7--3.7 m \\
Leaf area $A$        & 55--65 & 1320--1560 & 0.6--0.7 m$^{2}$ \\
Number of leaves $N$ & 65     & 1560       & 18--20 \\
Spikelets $c$        & 65--70 & 1560--1680 & $\sim$1000 \\
Fruit biomass $P$    & 125    & 3000       & 0.15--0.36 kg \\
\hline
\end{tabular}
\caption{Corn development timeline and typical final values from agronomic literature \cite{corn-growth-stages,corn-physiology}.}
\label{tab:corn-development}
\end{table}

%------------------------------------%
% OPTIMIZATION SETUP
%------------------------------------%
\subsection{Optimization Configuration}\label{subsec:opt-config}

The GA searches over the following bounds:
\begin{itemize}
\item Irrigation frequency: 100--700 hours (4--29 days between applications)
\item Irrigation amount: 0.5--5.0 inches per application
\item Fertilizer frequency: 700--2900 hours (29--121 days, i.e., 1--4 applications per season)
\item Fertilizer amount: 100--500 lbs per application
\end{itemize}

These bounds reflect practical constraints: irrigation systems have minimum application rates, and fertilizer is typically applied in a small number of large doses rather than continuously.

%------------------------------------%
% RESULTS
%------------------------------------%
\subsection{Optimization Results}\label{subsec:results}

Figure \ref{fig:ga-convergence} shows the GA convergence over 100 generations. The algorithm rapidly improves in early generations as poor strategies are eliminated, then converges to a near-optimal solution.

\begin{figure}[h]
\centering
\includegraphics[width=0.8\textwidth]{fig-eps-converted-to.pdf}
\caption{GA convergence showing best cost (solid) and mean parent cost (dashed) over generations. The algorithm converges within approximately 50 generations.}
\label{fig:ga-convergence}
\end{figure}

The optimal strategy identified by the GA is summarized in Table \ref{tab:optimal-strategy}. Notably, the algorithm discovers a strategy with less frequent but larger irrigation events and infrequent fertilizer applications---a pattern that minimizes cumulative divergence from expected nutrient levels given the model's delayed absorption dynamics.

\begin{table}[h]
\centering
\begin{tabular}{|c|c|c|}
\hline
Parameter & Optimal Value & Interpretation \\
\hline\hline
Irrigation frequency & 336 hours & Every 14 days \\
Irrigation amount & 0.23 inches & Per application \\
Fertilizer frequency & 2160 hours & Every 90 days ($\sim$2 applications) \\
Fertilizer amount & 175 lbs & Per application \\
\hline
\end{tabular}
\caption{Optimal irrigation and fertilization strategy identified by the GA.}
\label{tab:optimal-strategy}
\end{table}

Figure \ref{fig:state-trajectories} compares plant state trajectories under the optimal strategy versus a baseline ``typical'' strategy (weekly irrigation of 1 inch, monthly fertilization of 90 lbs). The optimal strategy achieves higher final fruit biomass despite using less total water.

\begin{figure}[h]
\centering
\includegraphics[width=0.9\textwidth]{fig-eps-converted-to.pdf}
\caption{State variable trajectories over the growing season. Solid lines: optimal strategy; dashed lines: typical strategy. The optimal strategy achieves 0.177 kg fruit biomass compared to 0.152 kg for the typical strategy.}
\label{fig:state-trajectories}
\end{figure}

Table \ref{tab:economic-comparison} provides an economic comparison. The optimal strategy achieves \$762/acre net revenue compared to \$614/acre for the typical strategy---a 24\% improvement---primarily through reduced input costs while maintaining competitive yields.

\begin{table}[h]
\centering
\begin{tabular}{|c|c|c|}
\hline
Metric & Optimal Strategy & Typical Strategy \\
\hline\hline
Final fruit biomass & 0.177 kg & 0.152 kg \\
Total irrigation & 2.07 inches & 17.4 inches \\
Total fertilizer & 350 lbs & 360 lbs \\
Crop value & \$981 & \$835 \\
Input costs & \$219 & \$221 \\
\textbf{Net revenue} & \textbf{\$762} & \textbf{\$614} \\
\hline
\end{tabular}
\caption{Economic comparison of optimal versus typical strategies. The optimal strategy achieves 24\% higher net revenue.}
\label{tab:economic-comparison}
\end{table}

Figure \ref{fig:nutrient-factors} shows the evolution of nutrient factors over the season. The optimal strategy maintains nutrient factors closer to 1 (no stress) throughout the critical grain-fill period, explaining its superior yield despite lower total inputs.

\begin{figure}[h]
\centering
\includegraphics[width=0.9\textwidth]{fig-eps-converted-to.pdf}
\caption{Nutrient factor evolution under optimal (solid) and typical (dashed) strategies. The optimal strategy maintains higher $\nu_{w}$ and $\nu_{f}$ during grain fill (hours 1500--2900).}
\label{fig:nutrient-factors}
\end{figure}

%======================================================================================%
% DISCUSSION
%======================================================================================%
\section{Discussion}\label{sec:discussion}

%------------------------------------%
% INTERPRETATION
%------------------------------------%
\subsection{Interpretation of Results}\label{subsec:interpretation}

The GA-optimized strategy differs from conventional wisdom in two notable ways. First, it applies less total irrigation (2.07 vs.\ 17.4 inches) but times applications to minimize cumulative divergence from expected levels. The model's delayed absorption mechanism means that frequent, small applications create sustained deviations as the FIR-convolved signal lags behind expectations. Less frequent, appropriately-sized applications better match the plant's metabolic timescales.

Second, the optimal fertilizer strategy applies larger doses less frequently. This aligns with the model's longer fertilizer absorption timescale ($\sigma_{f} = 300$ hours vs.\ $\sigma_{w} = 30$ hours for water), suggesting that fertilizer timing is less critical than total seasonal amount---consistent with agronomic practice where split applications are used primarily to reduce nutrient loss rather than optimize plant uptake timing.

The 24\% revenue improvement demonstrates the potential value of model-based optimization, though this result should be interpreted cautiously given model simplifications discussed below.

%------------------------------------%
% PARAMETER ESTIMATION
%------------------------------------%
\subsection{Parameter Estimation in Practice}\label{subsec:parameter-estimation}

The framework requires crop-specific parameters: growth rates, carrying capacities, metabolic timescales, and typical nutrient expectations. Several approaches could estimate these from data:

\begin{itemize}
\item \textbf{Growth curves:} Time-series imagery from field cameras or drones, processed with computer vision, could provide height and leaf area trajectories for fitting $a_{x}$ and $k_{x}$ parameters.
\item \textbf{Metabolic timescales:} The temporal spreads $\sigma$ could be estimated from controlled experiments varying input timing, or inferred from physiological literature on nutrient uptake rates.
\item \textbf{Typical expectations:} Regional agronomic recommendations provide baseline values for $w_{\text{typ}}$, $f_{\text{typ}}$, $T_{\text{typ}}$, and $R_{\text{typ}}$.
\end{itemize}

Physics-informed neural networks (PINNs) could jointly fit model parameters and approximate unknown functional forms in the dynamics, potentially relaxing some of the structural assumptions in Section \ref{sec:modeling}.

%------------------------------------%
% LIMITATIONS
%------------------------------------%
\subsection{Limitations and Extensions}\label{subsec:limitations}

Several model limitations suggest directions for future work:

\textbf{Growth model.} The logistic equation assumes symmetric growth around the inflection point. Richards growth \cite{richards1959} generalizes this with a shape parameter $\nu$:
\begin{equation}
\df{dx}{dt} = a_{x}x\left[ 1 - \left( \df{x}{k_{x}} \right)^{\nu} \right]
\label{eqn:richards-growth}
\end{equation}
where $\nu > 1$ produces steeper early growth (common in vegetative stages) and $\nu < 1$ produces steeper late growth.

\textbf{Absorption kernels.} Gaussian kernels are symmetric, but physiological absorption often exhibits fast activation followed by slow decay. Log-normal or Gamma kernels could better capture this asymmetry.

\textbf{Saturation.} The current model does not explicitly limit nutrient uptake---all applied inputs eventually affect the plant. In reality, excess application may be lost to runoff or leaching. Saturating nonlinearities in the absorption pathway would provide a more realistic response to over-application.

\textbf{Spatial heterogeneity.} The model treats a single representative plant. Field-scale optimization would need to account for spatial variation in soil properties, microclimate, and plant density.

\textbf{Stochastic weather.} The case study uses historical weather data. Robust optimization under weather uncertainty, or adaptive strategies that respond to observed conditions, could improve real-world performance.

%======================================================================================%
% CONCLUSION
%======================================================================================%
\section{Conclusion}\label{sec:conclusion}

This paper presented a coupled ODE model for crop growth that captures delayed nutrient absorption via FIR convolution and cumulative stress effects via EMA filtering. The model's time-varying growth rates and carrying capacities encode the intuition that plant development depends not just on current conditions but on the history of resource availability relative to physiological expectations.

Applied to corn optimization in Iowa, a genetic algorithm discovered irrigation and fertilizer strategies that achieve 24\% higher net revenue than conventional approaches by better aligning input timing with the plant's metabolic dynamics. The key insight is that under delayed, cumulative-effect dynamics, less can be more: strategic timing of inputs outperforms uniform application schedules.

The framework is generalizable to other crops through re-parameterization and offers a computationally tractable approach to input optimization. Future work will extend the model to handle weather uncertainty, incorporate spatial heterogeneity, and validate predictions against field trial data. 

%======================================================================================%
% SUPPLEMENTARY INFORMATION
%======================================================================================%
\backmatter
%\bmhead{Supplementary information}

%======================================================================================%
% ACKNOWLEDGEMENTS
%======================================================================================%
\bmhead{Acknowledgements}

This work has been partially supported by the UC Berkeley College of Engineering and the USDA AI Institute for Next Generation Food Systems (AIFS), USDA award number 2020-67021-32855.

%======================================================================================%
% DECLARATIONS
%======================================================================================%
\section*{Declarations}

\noindent\textbf{Competing Interests}
\noindent The authors declare that they have no known competing financial interests or personal relationships that could have appeared to influence the work reported in this paper.

\noindent\textbf{Code availability}
\noindent The source code used for this study is archived on Zenodo at [insert DOI link, e.g., doi.org] 

%======================================================================================%
% BIBLIOGRAPHY
%======================================================================================%
\bibliography{bibliography}

\end{document}
